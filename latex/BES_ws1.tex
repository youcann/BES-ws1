\documentclass[12pt,a4paper,noendnumber=true]{scrartcl}
%german umlauts and localization
\usepackage[utf8]{inputenc} 
\usepackage[T1]{fontenc}
\usepackage[ngerman]{babel}

%figures etc.
\usepackage[pdftex]{graphicx}
\usepackage{standalone} %externalize files for faster compilation
\usepackage{subcaption}
\usepackage{float}

%mathematical symbols
\usepackage{latexsym} % special symbols 
\usepackage{amsmath,amssymb,amsthm}
\usepackage{textcomp} % supports the Text Companion fonts, which provide many text symbols (such as baht, bullet, copyright, musicalnote, onequarter, section, and yen)
%fonts:
%\usepackage{txfonts}  %supplies virtual text roman fonts using Adobe Times % needs to be loaded AFTER amsmath (because otherwise \iint is defined twice)
\usepackage{mathrsfs}  % for script-like fonts in math mode
\usepackage{nicefrac} % nice fracs in text

%\usepackage{libertine}
%\usepackage[libertine]{newtxmath}
%\usepackage[sc]{mathpazo}
%\linespread{1.05}

%tables
\usepackage{tabularx}

%units
\usepackage{siunitx}

%tikz
\usepackage{tikz}
\usepackage{tikzscale}
\usepackage{pgfplots} 
\usepackage{pgfgantt}
\usepackage{pdflscape}
\usepackage[european]{circuitikz}
\pgfplotsset{compat=newest} 
\pgfplotsset{plot coordinates/math parser=false}
\usetikzlibrary{shapes.geometric}
\usetikzlibrary{arrows.meta}
\usetikzlibrary{arrows}
\usetikzlibrary{shapes.symbols,shadows}


%bib stuff
\usepackage[draft = false]{hyperref}
\usepackage{csquotes}
\usepackage[backend=biber,style=ieee]{biblatex}
%\addbibresource{bib.bib}


% gescheiter Abstand nach paragraph
\newcommand{\properparagraph}[1]{\paragraph{#1}\mbox{}\\}
\usepackage[parfill]{parskip}

% für die Auflistung von Vor- und Nachteilen in itemize-Umgebung
\newcommand\pro{\item[$+$]}
\newcommand\con{\item[$-$]}

%nice row vector
\newcommand{\rvect}[1]{\begin{bmatrix} #1 \end{bmatrix}}

%align multi pgfplots
\pgfplotsset{yticklabel style={text width=3em,align=right}}

%matlab listings
\usepackage{listings}
\usepackage{color} %red, green, blue, yellow, cyan, magenta, black, white
\definecolor{mygreen}{RGB}{28,172,0} % color values Red, Green, Blue
\definecolor{mylilas}{RGB}{170,55,241}
\lstset{language=Matlab,%
	%basicstyle=\color{red},
	breaklines=true,%
	morekeywords={matlab2tikz},
	keywordstyle=\color{blue},%
	morekeywords=[2]{1}, keywordstyle=[2]{\color{black}},
	identifierstyle=\color{black},%
	stringstyle=\color{mylilas},
	commentstyle=\color{mygreen},%
	showstringspaces=false,%without this there will be a symbol in the places where there is a space
	numbers=left,%
	numberstyle={\tiny \color{black}},% size of the numbers
	numbersep=9pt, % this defines how far the numbers are from the text
	emph=[1]{for,end,break},emphstyle=[1]\color{red}, %some words to emphasise
	%emph=[2]{word1,word2}, emphstyle=[2]{style},
	tabsize=2,
	basicstyle=\small,
	inputencoding=latin1    
}

\usetikzlibrary{external}
\tikzexternalize[optimize=false,prefix=tikz/] % activate!



\title{Workshop 1}
\subtitle{Bioelektrische Signale - Axel Loewe}
%\author{Marvin Noll}
\date{24.05.2019}


\begin{document}
\maketitle

\section{Aufgabe 1}

\begin{figure}[H]
	\centering
	\includegraphics[width=0.8\textwidth]{hh.tikz}
\end{figure}

\begin{subequations}
	\begin{align*}
		\alpha_n &= 0.01 \, \left( - \frac{V_m+55}{\exp\left(-\frac{V_m+55}{10}\right)-1} \right)\\
		\beta_n &= 0.125 \, \exp \left( - \frac{V_m+65}{80} \right)\\[3ex]
		\alpha_m &= 0.1 \, \left( - \frac{V_m+40}{\exp(-\frac{V_m+40}{10})-1} \right) \\
		\beta_m &= 4 \, \exp \left( - \frac{V_m+65}{18} \right)\\[3ex]
		\alpha_h &= 0.07 \, \exp \left( - \frac{V_m+65}{20} \right)\\
		\beta_h &= \frac{1}{\exp \left( - \frac{V_m+35}{10} \right)+1}
	\end{align*}
\end{subequations}

\begin{equation*}
	V_m= \rvect{-100 & -90 & \ldots & -40 & \ldots & 30 & 40}
\end{equation*}

\begin{equation*}
	\alpha_m(V_m)\vert_{V_m=\SI{-40}{\milli\volt}} = \text{undef.}
\end{equation*}

\begin{equation*}
	\alpha_m(\SI{-40}{\milli\volt}) := \lim\limits_{V_m \to \SI{-40}{\milli\volt}} = 1
\end{equation*}

\begin{figure}[H]
	\begin{subfigure}[t]{\textwidth}
		\centering
		\includegraphics[width=\textwidth,height=0.35\textwidth]{plots/a1_n.tikz}
	\end{subfigure}
	\\
	\begin{subfigure}[t]{\textwidth}
		\centering
		\includegraphics[width=\textwidth,height=0.35\textwidth]{plots/a1_m.tikz}
	\end{subfigure}
	\\
	\begin{subfigure}[t]{\textwidth}
		\centering
		\includegraphics[width=\textwidth,height=0.35\textwidth]{plots/a1_h.tikz}
	\end{subfigure}
	\caption{Übergangsraten $\alpha$ und $\beta$ für $n$, $m$ und $h$}
\end{figure}

\begin{lstlisting}
clampVoltages = linspace(-100,40,15);

nVoltages = length(clampVoltages); % Anzahl an Clampspannungen
alpha_n   = zeros(nVoltages, 1); 
beta_n    = zeros(nVoltages, 1);
alpha_m   = zeros(nVoltages, 1);
beta_m    = zeros(nVoltages, 1);
alpha_h   = zeros(nVoltages, 1);
beta_h    = zeros(nVoltages, 1);

for iVoltage = 1:nVoltages
	alpha_n(iVoltage) = 0.01*(-(clampVoltages(iVoltage)+55)/(exp(-(clampVoltages(iVoltage)+55)/(10))-1));
	beta_n(iVoltage)  = 0.125*exp(-(clampVoltages(iVoltage)+65)/(80));
	
	alpha_m(iVoltage) = 0.1*(-(clampVoltages(iVoltage)+40)/(exp(-(clampVoltages(iVoltage)+40)/(10))-1));        
	beta_m(iVoltage)  = 4*exp(-(clampVoltages(iVoltage)+65)/(18));
	
	alpha_h(iVoltage) = 0.07*exp(-(clampVoltages(iVoltage)+65)/(20));
	beta_h(iVoltage)  = 1/(exp(-(clampVoltages(iVoltage)+35)/(10))+1);
end
alpha_n(isnan(alpha_n))=1;
beta_n(isnan(beta_n))  =1;
alpha_m(isnan(alpha_m))=1;
beta_m(isnan(beta_m))  =1;
alpha_h(isnan(alpha_h))=1;
beta_h(isnan(beta_h))  =1;
\end{lstlisting}






\section{Aufgabe 2}

\begin{subequations}
	\begin{align*}
	I_K &= \bar{G}_K \cdot n^4 \; (V_m-E_K) \\
	I_{Na} &= \bar{G}_{Na} \cdot m^3 \; h \; (V_m-E_{Na}) 
	\end{align*}
\end{subequations}

\begin{equation*}
	\begin{array}{r@{\ }c@{\ }l}
	\dot{n}= & \alpha_n \; (1-n) &- \, \beta_n \cdot n \\
	\dot{m}= & \alpha_m \; (1-m) &- \, \beta_m \cdot m \\
	\dot{h}= & \alpha_h \; (1-h) &- \, \beta_h \cdot h 
	\end{array}
\end{equation*}

\begin{subequations}
	\begin{align*}
	n_{i+1} &= n_i + \Delta t \cdot \dot{n}_i \\ 
	m_{i+1} &= m_i + \Delta t \cdot \dot{m}_i \\
	h_{i+1} &= h_i + \Delta t \cdot \dot{h}_i 
	\end{align*}
\end{subequations}

\begin{figure}[H]
	\centering
	\includegraphics[width=0.6\textwidth]{plots/a2_Vm.tikz}
\end{figure}

\newpage

\begin{figure}[H]
	\begin{subfigure}[t]{\textwidth}
		\centering
		\includegraphics[width=\textwidth,height=0.45\textwidth]{plots/a2_K_1.tikz}
	\end{subfigure}
	\\
	\begin{subfigure}[t]{\textwidth}
		\centering
		\includegraphics[width=\textwidth,height=0.45\textwidth]{plots/a2_K_2.tikz}
	\end{subfigure}
	\\
	\begin{subfigure}[t]{\textwidth}
		\centering
		\includegraphics[width=\textwidth,height=0.45\textwidth]{plots/a2_K_3.tikz}
	\end{subfigure}
	\caption{Kalium}
\end{figure}

\begin{figure}[H]
	\begin{subfigure}[t]{\textwidth}
		\centering
		\includegraphics[width=\textwidth,height=0.45\textwidth]{plots/a2_Na_1.tikz}
	\end{subfigure}
	\\
	\begin{subfigure}[t]{\textwidth}
		\centering
		\includegraphics[width=\textwidth,height=0.45\textwidth]{plots/a2_Na_2.tikz}
	\end{subfigure}
	\\
	\begin{subfigure}[t]{\textwidth}
		\centering
		\includegraphics[width=\textwidth,height=0.45\textwidth]{plots/a2_Na_3.tikz}
	\end{subfigure}
	\caption{Natrium}
\end{figure}

\begin{lstlisting}
% Konstanten
G_K_max    = 36;     % Maximale Kalium-Leitfaehigkeit (mS)
G_Na_max   = 120;    % Maximale Natrium-Leitfaehigkeit (mS)

% Spannungen (Nernstspannungen koennen als konstant betrachtet werden)
E_K        =  -88;   % Nernstsspannung für Kalium (mV) 
E_Na       =   50;   % Nernstsspannung für Natrium (mV)
VpreStep   =  -65;   % Spannung vor Sprung (mV)
VpostStep  = -110;   % Spannung nach Sprung (mV)

clampVoltages = -100:10:40; % Protokoll fuer Voltageclamp (mV)

% Initialisiserung
n_0        = 0.32;
m_0        = 0.053;
h_0        = 0.6;

% Zeit
deltat     = 0.0005; %Zeitschritt (ms)
tend       = 10;     %Ende der Berechnung (ms)
timesteps  = 0:deltat:tend;

% Preallokation der Matrix fuer die Raten, Gates, Stroeme, und Spannung
nVoltages  = length(clampVoltages);
nTimesteps = length(timesteps);
alpha_n    = zeros(nVoltages, nTimesteps); 
beta_n     = zeros(nVoltages, nTimesteps); 
n          = zeros(nVoltages, nTimesteps); 
ndot       = zeros(nVoltages, nTimesteps);
I_K        = zeros(nVoltages, nTimesteps); 
alpha_m    = zeros(nVoltages, nTimesteps); 
beta_m     = zeros(nVoltages, nTimesteps); 
m          = zeros(nVoltages, nTimesteps); 
mdot       = zeros(nVoltages, nTimesteps); 
alpha_h    = zeros(nVoltages, nTimesteps); 
beta_h     = zeros(nVoltages, nTimesteps); 
h          = zeros(nVoltages, nTimesteps); 
hdot       = zeros(nVoltages, nTimesteps); 
I_Na       = zeros(nVoltages, nTimesteps);
%Startwerte
n(:,1)     = n_0*ones(1,nVoltages);
m(:,1)     = m_0*ones(1,nVoltages);
h(:,1)     = h_0*ones(1,nVoltages);

%% Simulation
% Berechne Raten, Gates (Euler-1-Schritt), I_K und I_Na
for iVoltage=1:nVoltages
	for iTimestep=1:nTimesteps
		%Vm
		if (timesteps(iTimestep)<=0.3)
			Vm_current=VpreStep;
		elseif (timesteps(iTimestep)>0.3 && timesteps(iTimestep)<=7.3)
			Vm_current=clampVoltages(iVoltage);    
		elseif (timesteps(iTimestep)>7.3 && timesteps(iTimestep)<=10)
			Vm_current=VpostStep;   
		else
			Vm_current=NaN;   
		end
		
		% Kalium
		I_K(iVoltage,iTimestep) = G_K_max * n(iVoltage,iTimestep).^4 * (Vm_current-E_K);
		
		% Natrium
		I_Na(iVoltage,iTimestep) = G_Na_max * m(iVoltage,iTimestep).^3 * h(iVoltage,iTimestep) * (Vm_current-E_Na);
		
		%Uebergangsraten
		alpha_n(iVoltage,iTimestep) = 0.01*(-(Vm_current+55)/(exp(-(Vm_current+55)/(10))-1));
		if(isnan(alpha_n(iVoltage,iTimestep))) alpha_n(iVoltage,iTimestep) = 1; end
		beta_n(iVoltage,iTimestep) = 0.125*exp(-(Vm_current+65)/(80));
		if(isnan(beta_n(iVoltage,iTimestep))) beta_n(iVoltage,iTimestep) = 1; end
		
		alpha_m(iVoltage,iTimestep) = 0.1*(-(Vm_current+40)/(exp(-(Vm_current+40)/(10))-1));
		if(isnan(alpha_m(iVoltage,iTimestep))) alpha_m(iVoltage,iTimestep) = 1; end
		beta_m(iVoltage,iTimestep) = 4*exp(-(Vm_current+65)/(18));
		if(isnan(beta_m(iVoltage,iTimestep))) beta_m(iVoltage,iTimestep) = 1; end
		
		alpha_h(iVoltage,iTimestep) = 0.07*exp(-(Vm_current+65)/(20));
		if(isnan(alpha_h(iVoltage,iTimestep))) alpha_h(iVoltage,iTimestep) = 1; end
		beta_h(iVoltage,iTimestep) = 1/(exp(-(Vm_current+35)/(10))+1);
		if(isnan(beta_h(iVoltage,iTimestep))) beta_h(iVoltage,iTimestep) = 1; end
		
		%nmh dots
		ndot(iVoltage,iTimestep) = alpha_n(iVoltage,iTimestep) .* (1-n(iVoltage,iTimestep)) - beta_n(iVoltage,iTimestep) .* n(iVoltage,iTimestep);
		mdot(iVoltage,iTimestep) = alpha_m(iVoltage,iTimestep) .* (1-m(iVoltage,iTimestep)) - beta_m(iVoltage,iTimestep) .* m(iVoltage,iTimestep);
		hdot(iVoltage,iTimestep) = alpha_h(iVoltage,iTimestep) .* (1-h(iVoltage,iTimestep)) - beta_h(iVoltage,iTimestep) .* h(iVoltage,iTimestep);
		
		if(iTimestep < nTimesteps)
			n(iVoltage,iTimestep+1) = deltat * ndot(iVoltage,iTimestep) + n(iVoltage,iTimestep);
			m(iVoltage,iTimestep+1) = deltat * mdot(iVoltage,iTimestep) + m(iVoltage,iTimestep);
			h(iVoltage,iTimestep+1) = deltat * hdot(iVoltage,iTimestep) + h(iVoltage,iTimestep);
		end    
	end
end

\end{lstlisting}


\end{document}

